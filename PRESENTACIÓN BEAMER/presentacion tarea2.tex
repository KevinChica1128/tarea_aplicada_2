\documentclass[12pt]{beamer}
\usetheme{CambridgeUS}
\usepackage[utf8]{inputenc}
\usepackage[spanish]{babel}
\usepackage{amsmath}
\usepackage{amsfonts}
\usepackage{amssymb}
\usepackage{graphicx}
\author{Kevin García - Alejandro Vargas}
\title{El modelo de regresión lineal simple}
%\setbeamercovered{transparent} 
%\setbeamertemplate{navigation symbols}{} 
%\logo{} 
%\institute{} 
%\date{} 
%\subject{} 
\begin{document}

\begin{frame}
\titlepage
\end{frame}

%\begin{frame}
%\tableofcontents
%\end{frame}
\begin{frame}
\frametitle{Selección de la base de datos}
~\\La base de datos 'empleados1' cuenta con información sobre la edad, la estatura y el peso
de 99 personas, 12 de ellas mujeres. De esas 99 personas debíamos seleccionar una muestra de 24, fijando las 12 mujeres, por lo tanto, nos quedaron 87 hombres de los cuales debíamos seleccionar 12. La selección se hizo con un sample de R, el cuál me arroja 12 números aleatorios entre 1 y 87, esos 12 números generados fueron nuestros hombres seleccionados.
\end{frame}

\begin{frame}
\frametitle{Punto 1:Modelo lineal simple}
~\\ El modelo lineal ajustado para la variable peso con la variable predictora 'estatura' fue:
~\\ $$Peso=-99.0330+0.9778 Estatura$$

\end{frame}

\begin{frame}
\frametitle{Punto 2: Bondad del modelo e interpretaciones}
\begin{itemize}
\item $R^2=0.5754$: El 57.54\% de la variabilidad total de la variable Y:'Peso' es explicada por la variable X:'Estatura'
\item $\beta_{0}=-99.0330$:
\item $\beta_{1}=0.9778$: Cuando la variable 'Estatura', aumenta en una unidad (1 centímetro), se espera que el 'Peso' de la persona aumente en 0.9778 kg.
\item p-valor $\beta_{0}=0.00425$: 
\item p-valor $\beta_{1}=0.0000174$:
\end{itemize}
\end{frame}

\begin{frame}
\frametitle{Punto 3:Intervalos de confianza para $\beta_{0}$ y $\beta_{1}$}
\begin{itemize}
\item $\beta_{0}$: (-163.456 ; -34.610)  ; El verdadero valor de $\beta_{0}$ está entre -163.456 y -34.610 con una confianza del 95\%
\item $\beta_{1}$: (0.6063928 ; 1.3492072) ; El verdadero valor de $\beta_{1}$ está entre 0.6064 y 1.3492 con una confianza del 95\%
\end{itemize}

\end{frame}
\begin{frame}
\frametitle{Punto 4:Inclusión de la variable 'Sexo' al modelo}
~\\ Para incluir la variable sexo al modelo, recodificamos la variable en términos binarios, la cual tomaba el valor 0 cuando es mujer y 1 cuando es hombre, el modelo ajustado incluyendo la nueva variable recodificada fue el siguiente:
~\\ $$Peso=-84.4593+0.8778 Altura +  5.4792 Sexo $$
\end{frame}
\begin{frame}
\frametitle{Punto 5:Comparación de modelos}

\end{frame}

\begin{frame}
\frametitle{Punto 6:Inclusión de la variable 'Edad' en el modelo}
~\\ El modelo ajustado incluyendo la variable edad, es el siguiente:
~\\ $$Peso=-102.6256+0.8759 Altura +0.9867 Edad +4.4248 Sexo $$
\end{frame}




\end{document}